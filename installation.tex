\section[Installation de Flutter]{Installation de Flutter \includegraphics[height=15pt]{figures-logos/flutter.pdf}}

\par $\bullet$ \textbf{NE PAS INSTALLER DART SEPAREMENT DE FLUTTER. SI VOUS INSTALLEZ FLUTTER, IL SE CHARGE D'INSTALLER DART}
\par $\bullet$  \href{https://docs.flutter.dev/get-started/install}{Cliquez ici} pour le guide officiel d'installation, sur lequel nous nous baserons.

Choisissez la section correspondant à votre système d'exploitation:

\begin{figure}[!h]
  \centering
  \begin{minipage}{0.32\textwidth}
    \centering
    \hyperref[sec:installation_linux]{\includegraphics[scale=0.05]{Images_formation/LinuxLogo.pdf}}
    \caption*{\underline{Linux}: \textsc{section~\ref{sec:installation_linux}}}
  \end{minipage}
  \begin{minipage}{0.32\textwidth}
    \centering
    \hyperref[sec:installation_windows]{\includegraphics[scale=0.020]{Images_formation/WindowsLogo.pdf}}
    \caption*{\underline{Windows}: \textsc{section~\ref{sec:installation_windows}}}
  \end{minipage}
  \begin{minipage}{0.32\textwidth}
    \centering
    \hyperref[sec:installation_macos]{\includegraphics[scale=0.025]{Images_formation/MacosLogo.pdf}}
    \caption*{\underline{MacOS}: \textsc{section~\ref{sec:installation_macos}}}
  \end{minipage}
\end{figure}
\subsection[Installation Linux]{Installation \linux{}\label{sec:installation_linux}}

\par Etant donné la diversité des distributions Linux, nous n'allons ici couvrir que la plus commune : \textbf{Ubuntu}, plus précisément \textbf{Ubuntu 22.04}. \\
Si vous utilisez une autre distribution, l'installation devrait  être identique en majeure partie.

\begin{enumerate}
  \item Pour commencer, ouvrez votre terminal et vérifiez que ces différents outils sont installés: \texttt{bash}, \texttt{file}, \texttt{mkdir}, \texttt{rm}, \texttt{which} grâce à la commande

    \vspace{0.3cm}
    \begin{lstlisting}
                which bash file mkdir rm which
    \end{lstlisting}
    \vspace{0.2cm}

  \item Mettez à jour la liste des packages:

    \vspace{0.3cm}
    \begin{lstlisting}
                sudo apt-get update -y && sudo apt-get upgrade -y
    \end{lstlisting}
    \vspace{0.2cm}

  \item Installez les packages suivants: \texttt{curl}, \texttt{git}, \texttt{unzip}, \texttt{xz-utils}, \texttt{zip}, \texttt{libglu1-mesa}

    \vspace{0.3cm}
    \begin{lstlisting}
                sudo apt-get install -y curl git unzip xz-utils zip libglu1-mesa
    \end{lstlisting}
    \vspace{0.2cm}

  \item Pour les applications Android, nous avons besoin d'\texttt{Android Studio}. Installez donc donc les packages prérequis suivants:

    \vspace{0.3cm}
    \begin{lstlisting}
                sudo apt-get install \
                libc6:amd64 libstdc++6:amd64 \
                libbz2-1.0:amd64 libncurses5:amd64
    \end{lstlisting}
    \vspace{0.2cm}

  \item Cliquez \href{https://developer.android.com/studio?hl=fr}{ici} pour télecharger la dernière version d'\texttt{Android Studio}.\\

  \item Décompressez le fichier .zip téléchargé vers un emplacement approprié pour vos applications (par exemple, dans /usr/local/ pour votre profil utilisateur ou /opt/ pour les utilisateurs partagés).\\

  \item Pour lancer \texttt{Android Studio}, accédez au répertoire android-studio/bin/, puis exécutez:

    \vspace{0.3cm}
    \begin{lstlisting}
                studio.sh
    \end{lstlisting}
    \vspace{0.2cm}

  \item Indiquez si vous souhaitez importer les anciens paramètres Android Studio, puis cliquez sur \textbf{OK}.\\

  \item Suivez l'\textbf{assistant de configuration} d'\texttt{Android Studio} (ce qui implique de télécharger les composants du SDK Android requis pour le développement).\\

  \item Ouvrez un terminal, et entrez:

    SURTOUT PAS SNAP
    \begin{lstlisting}
                sudo snap install flutter --classic
    \end{lstlisting}

  \item Tapez \textbf{flutter} pour finir l'installation.

\end{enumerate}

\bigskip

\hyperref[sec:suite_installation]{Pour passer à la suite}
\newpage

\subsection[Installation Windows]{Installation windows \label{sec:installation_windows}}
% \subsubsection{Via Visual Studio Code}
%     \begin{itemize}
%         \item \textbf{Attention, vous devez avoir Windows 10 ou autre Windows plus récent !}

%         \item Choisissez l'installation Windows, puis Android.

%         \item Installez Visual Studio Code, un éditeur de code fort pratique, grâce à \href{https://code.visualstudio.com/docs/setup/windows}{ce guide}.

%         \item Lancez VSCode, et dans la barre à gauche, cliquez sur "\textit{Extensions}.

%         \item Cherchez "\textit{flutter}", et installez la 1ère extension de la liste (elle s'appelle juste "Flutter".

%         \item Faites CTRL + SHIFT + P, et tapez dans la barre "\textit{flutter}". Cliquez sur "\textit{Flutter : New Project}.

%         \item VSCode vous informe qu'il ne trouve le Flutter SDK : cliquez sur "\textit{Download SDK}".

%         \item Votre explorateur de dossiers s'ouvre, choisissez un dossier ou installer Flutter. \textbf{N'installez pas dans un dossier cotnenant des espaces ou des charactèrs spéciaux dans son nom, ni un dossier avec des autorisations d'accès.} \\
%         \textit{C:$\backslash$Users$\backslash$[votre\_nom\_utilisateur]} est un bon choix.

%     \end{itemize}

\subsubsection{Via téléchargement}

\begin{enumerate}

  \item \textbf{Attention, vous devez avoir Windows 10 ou plus récent !}

  \item Choisissez l'installation Windows, puis Android.

  \item Téléchargez le fichier \textbf{"flutter\_window\_[version].zip"}.

  \item Déplacez le dans "\textbf{C:$\backslash$Users$\backslash$[votre\_nom\_utilisateur]}".

  \item Créez dans \textbf{"C:$\backslash$Users$\backslash$[votre\_nom\_utilisateur]"} un dossier \textbf{"dev"}, et déplacez-y le fichier .zip. Extrayez les fichiers. Normalement un dossier du même nom a été créé, avec à l'intérieur un dossier "\textit{flutter"}.

  \item (Redéplacez ce dernier dans \textbf{"C:$\backslash$Users$\backslash$[votre\_nom\_utilisateur]"}).

  \item Rendez vous dans le dossier "\textit{flutter}" puis "\textit{bin}".

  \item Copiez le chemin de fichiers au dessus.

  \item Tapez "var" dans la barre de recherche Windows, et cliquez sur le 1er résulat \textbf{"Modifier les variabled d'environnement du système"}.

  \item Allez sur \textbf{"Paramètres systèmes avancés"}, puis \textbf{"Variables d'environnement"}.

  \item Dans la partie \textbf{"Variables utilisateur pour [votre\_nom\_utilisateur]"}, cliquez sur la variable \textbf{"Path"}, puis "Modifier".

  \item Cliquez sur "Nouveau", et collez le chemin de fichier que vous avez copié plus tôt. Puis cliquez sur "Déplacer vers le haut".

  \item Cliquez sur "OK" dans chaque fenêtre. Si vous vous contenter de les fermer avec la croix, les changements ne seront pas pris en compte.

  \item Tapez dans la barre de recherche Windows "cmd", et cliquez sur \textbf{"Invité de commandes"}.

  \item Entrez la commande \textbf{"flutter --version"}. Si vous voyez bien la version de Flutter, vous avez correctement installé Flutter.
    % Attention, ce n'est pas fini ! Il y a encore d'autres choses à installer !

\end{enumerate}

\subsection[Installation Mac]{Installation macos \label{sec:installation_macos}}

\subsection{En plus\label{sec:suite_installation}}

