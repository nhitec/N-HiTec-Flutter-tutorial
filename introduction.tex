\section{Introduction}\label{introduction}

\subsection{Prélude}
\par Vous avez toujours rêvé de créer une application mobile innovante, mais les défis techniques vous ont freiné ? Qu'il s'agisse de concevoir une interface élégante, d'intégrer des fonctionnalités fluides ou de coder de manière robuste et maintenable, le développement d'applications mobiles peut sembler être un parcours semé d'embûches. Sans oublier les tests rigoureux nécessaires pour garantir une expérience utilisateur optimale sur une multitude de dispositifs.

\par C'est là que \flutter{} entre en jeu, offrant une solution révolutionnaire qui transforme ces défis en opportunités ! Imaginez un framework\footnote{Un \href{https://fr.wikipedia.org/wiki/Framework}{\texttt{framework}} est un ensemble cohérent d'outils et de bibliothèques fournissant une structure et des fonctionnalités de base pour le développement d'applications.} qui vous permet de créer des applications esthétiques, performantes et multi-plateformes avec une seule base de code. \flutter{} n'est pas seulement un outil, c'est une porte ouverte vers l'innovation dans le monde du développement mobile.

\subsection{Prérequis}
\par Il n'y a pas de prérequis stricts, à part un ordinateur évidemment. Toutefois, des connaissances de base en programmation, en particulier dans des langages orientés objet comme \java{} ou \cplusplus{}, sont un atout considérable.

\subsection{Éditeur de code (IDE)}
Vous pouvez créer des applications avec \flutter{} à l'aide de n'importe quel éditeur de texte ou environnement de développement intégré (IDE) combiné aux outils de ligne de commande.

L'utilisation d'un IDE avec une extension ou un plugin \flutter{} permet la complétion du code, la coloration syntaxique, l'assistance à l'édition des widgets, le débogage et d'autres fonctionnalités.

Pour cette formation, nous recommandons\footnote{Utilisez VS Code ou conséquences \cat{}} vivement l'utilisation de \textbf{VS Code} car cela simplifie pas mal de choses telle que l'installation de \texttt{Flutter SDK} que nous verrons un peu plus tard.

\newpage
\subsection{Qu'est ce que \flutter{} ? }

\begin{floatingfigure}[r]{0.25\textwidth}
  \centering
  \includegraphics[width=0.3\textwidth]{Images_formation/FlutterLogo.png}
\end{floatingfigure}

\par \flutter{} est un framework open-source développé par Google pour la création d'interfaces utilisateur (UI). Il permet de concevoir des applications natives multiplateformes pour mobile, web et desktop à partir d'une seule base de code. Google a  conçu ce framework afin de simplifier le développement d'applications mobiles, même pour ceux qui ne possèdent pas de vastes connaissances en programmation.

\subsection{Les avantages de \flutter{}}

\begin{itemize}

  \item[$\sbullet$] \textbf{Développement multiplateforme} : Avec un seul code, vous pouvez développer une application compatible avec Windows, Linux, macOS, Android, iOS et le web. Vous n'aurez que rarement à vous soucier des problèmes de compatibilité.\\

  \item[$\sbullet$] \textbf{Approche orientée objet} : \flutter{} utilise une approche orientée objet, intuitive pour le développement d'applications. Si vous n'êtes pas familier avec cette approche, vous pouvez regarder \href{https://www.youtube.com/watch?v=gABYMZbfGok&ab_channel=BandedeCodeurs}{cette vidéo} pour en avoir une idée. Mais ne vous inquiétez pas, il n'y a pas besoin d'être un crack en OOP pour suivre cette formation et développer de jolies applications !\\

  \item[$\sbullet$] \textbf{Langage \dart{}} : Flutter utilise son propre langage de programmation, Dart, conçu pour être facile à apprendre et à utiliser.\\

  \item[$\sbullet$] \textbf{Système de widgets} : Le système des widgets permet de développer facilement des interfaces, de la plus simple à la plus complexe. Chaque élément de l'interface est un widget que vous pouvez personnaliser et composer.\\

  \item[$\sbullet$] \textbf{Richesse des packages} : Si vous avez une idée de fonctionnalité, vous pouvez rechercher parmi les nombreux packages disponibles pour voir si un package existant répond déjà à vos besoins, ce qui vous fait gagner du temps.\\

  \item[$\sbullet$] \textbf{Documentation et support} : Flutter offre une documentation claire et complète, des tutoriels vidéo et même \href{https://www.youtube.com/@flutterdev/videos}{une chaîne YouTube} dédiée. Vous pouvez accéder à la documentation officielle \href{https://flutter.dev/docs}{ici}.\\

  \item[$\sbullet$] \textbf{Testing en temps réel} : Vous pouvez tester votre application en temps réel directement sur votre ordinateur, que ce soit en tant que programme, site web ou même sur une simulation de téléphone grâce à \texttt{Android Studio}.\\

  \item[$\sbullet$] \textbf{Hot reload} : \flutter{} facilite le débogage avec le hot reload, une fonctionnalité qui permet d'appliquer les changements dans votre code sans devoir fermer et relancer votre application.\\

  \item[$\sbullet$] \textbf{Conversion en application mobile} : Vous pouvez facilement convertir votre code en une application utilisable sur votre téléphone.\\

\end{itemize}

